\documentclass[conference]{IEEEtran}
\IEEEoverridecommandlockouts
% The preceding line is only needed to identify funding in the first footnote. If that is unneeded, please comment it out.
\usepackage{cite}
\usepackage{amsmath,amssymb,amsfonts}
\usepackage{algorithmic}
\usepackage{graphicx}
\usepackage{textcomp}
\usepackage{xcolor}
\def\BibTeX{{\rm B\kern-.05em{\sc i\kern-.025em b}\kern-.08em
    T\kern-.1667em\lower.7ex\hbox{E}\kern-.125emX}}
\begin{document}

\title{Conference Paper Title*\\
{\footnotesize \textsuperscript{*}Note: Sub-titles are not captured in Xplore and
should not be used}
\thanks{Identify applicable funding agency here. If none, delete this.}
}


\maketitle

\begin{abstract}
The paper has discussed about the racial matter in the UK where the black people were facing the racism issues. The justice system and police force were questioning about their racial compariosn with the black people. The paper discussed the arreset rate and how th number were showing the ethnicity people arrest rate.
\end{abstract}

\begin{IEEEkeywords}
component, formatting, style, styling, insert
\end{IEEEkeywords}

\section{Introduction}
There are a number of events that are describing the police practices in UK. The practices are covering the violence against black people in UK. This has been a concerning matter in UK also where the police violence is taking place against the black people. The concerning matter here is arising the discrimination in the ethnicity where black people are being targeted to physical violence by the police force. The data is going to prove the evidence about the matter and the dataset is going to discuss how the police forces are triggering the black people in this case and arrest rates are increasing day by day.



\section{discussion}
The people in the UK are believing that the criminal justice system and police forces are biased while the racial protests are taking place in various areas. There is various poll have been organized to show the protest against the justice system and police forces. The major amount of people is hoping the matter as a concerning one that they are feeling distracted while the fact is coming into police force. Hence, the number of eight people out of 10 people are fearing about the biasness in ethnic group.
 There are 65 percent of people who are agreeing in this fact. Hence, the poll is also finding the fact that 4 out of 10 ethnics are experiencing the racial violence in recent times. The racial abusing matter has bene coming in the front in today’s world while the real face of the police force and justice system has been revealed. The black people who are residing in the UK are not feeling safe any way after watching recent conditions. The conditions are provoking them to run away from the country as the biasness of police has been increasing day by day. The black citizens were fearing that if they are finding themselves in any trouble they will not come to the police for help. The concerning factor has been rising that the police forces and justice system are showing biasness and ethnic groups were feeling unsafe in this condition. The black people are considering as the citizen in this country where they are noticing the biasness in this category. The concerning factor is indicating the fact that the police forces in the UK are measuring the biasness in this process. The biasness has been taking place in everywhere while the justice system is also giving the priority to the white people. 
\section{overview}
The police force in England and Wales were facing various inquiry as per their action against the black people. They were discriminating the ethnic minors and their force were going against the black people. The cases have been investigated for this case while the racial discrimination patterns were showing the issues against the black people. Furthermore, questions are arousing against the UK police squad as they were preferring black people to be arrested more. The protests are occurring in various areas where the black people are raising their voice against the police forces in the UK. Hence, the police force was still denying the fact that they were supporting racial issues while the evidences were proving that they were somehow targeting the black people in this case.

\section{arrest rate}
The above chart has been showing the arrest rate of different ethnics in this condition. The arrest rate has been indicating the number of arrests. The chart is discussing that the arrest rate year wise as how the different age group ethnics are falling for the threat. The data has been provided above to indicate the rate. Hence, the ethnicity is showing that mostly Asian people are suffering the biasness of the police force where the justice system is also going against them.
\section{arrest by ethnicity}
The above dataset is providing the discussion about the racism against black people. The data is showing that the area wise arrest rate is indicating the area wise measurement in this case. The Derbyshire area is showing that the rate per 1000 arrest rate in this process where the total number of arrests were showing that 10,383 people in all the ethnic and non-ethnic basis people. The arrest rate for Asian people is indicating the number as 17 and the total number of arrests were 640. The arrests rate for the black people is indicating the rate as 52 and total number of black people arrested was 522. The simple data is showing the racism against the black people. The overall arrests rate was 10 while only the black people has been arrested more times. 
This is measuring the roadmap in this process to show the issue against the black people. Hence, the arrests rate in Cumbria is showing the number as 15 while the total number of arrest rate was 7512. The arrests rate of Asian people was showing as 19 and the total number of 55 Asian people has been arrested. The black people was showing that the arrest rate in this process as 15 while the total number of 522 has been arrested in this case. The Chesire area has denoted that the arrest rate in this area was 17 while a number of 9703 people has been arrested. Hence, the arrest rate in Asian people has mentioned the rate as 12 and a total of 159 people has been arrested. In the term of black people, the arrest rate was 75 and a number of 244 people has been arrested. This data has been showed that how UK police were showing their racist behavior against black people. 
\section{observation}
The datasheet has been mentioned in this case where the black people were the one who are facing serious issue in this condition. Hence, the data has bene observed in this process to show the comparison of justice system. The data has been provided in this case to showcase the racism context against the black people who are residing in the UK. The ethnic people are following the poll result and deciding to protest against the police force. The data has been showing the racist activity while the matter is coming to safety. The ethnic people are not feeling safe in these places.
\section{conclusion}
The above discussion has addressed the concerning matter of racism against the black people who were residing in the UK. The dataset has also been provided to support the discussion and the major condition has been appearing in this purpose where the average arrest rate and ethnicity number has been showed. The data has also been suggested how the racial concern has been affecting them in justice system and the dataset that has been attached in this paper that were considering the comparison with the ethnic people and white people. This has been a concerning factor in this process where the ethnic people were noticing the racism against the justice system and entire police force. The police force was generally triggering to the black people and they were now feeling unsafe in this condition. The poll has been conducted to showcase the difference among the people in UK and the poll results were showing that major number of black people were agreeing with the fact that they were facing racism in judicial system. 




\section*{References}



\begin{thebibliography}{00}
\bibitem{b1} [1]S. Kolstoe, D. Shanahan and J. Wisely, "Should research ethics committees police reporting bias?", BMJ, p. j1501, 2017. Available: 10.1136/bmj.j1501.
\bibitem{b2} [2]K. Murphy, R. Cramer, K. Waymire and J. Barkworth, "Police Bias, Social Identity, and Minority Groups: A Social Psychological Understanding of Cooperation with Police", Justice Quarterly, vol. 35, no. 6, pp. 1105-1130, 2017. Available: 10.1080/07418825.2017.1357742.
\bibitem{b3} [3]P. Brantingham, M. Valasik and G. Mohler, "Does Predictive Policing Lead to Biased Arrests? Results From a Randomized Controlled Trial", Statistics and Public Policy, vol. 5, no. 1, pp. 1-6, 2018. Available: 10.1080/2330443x.2018.1438940.
\bibitem{b4}[4]J. Jones, "Killing Fields: Explaining Police Violence against Persons of Color", Journal of Social Issues, vol. 73, no. 4, pp. 872-883, 2017. Available: 10.1111/josi.12252.
\bibitem{b5}[5]M. Dai, "Training police for procedural justice: An evaluation of officer attitudes, citizen attitudes, and police-citizen interactions", The Police Journal: Theory, Practice and Principles, pp. 0032258X2096079, 2020. Available: 10.1177/0032258x20960791.
\bibitem{b6}[6]M. Noya-Rabelo, "Unconscious BIAS: we are more biased than we think!", Journal of Evidence-Based Healthcare, vol. 1, no. 1, p. 40, 2019. Available: 10.17267/2675-021xevidence.v1i1.2262.

\end{thebibliography}


\end{document}
